\section{Simulation input}

Leading spaces and comments at the end of each line are omitted. Empty lines are skipped, 
and case is not important except in file names (i.e. framework and molecule names).

\subsection*{Simulation types}
\begin{itemize}
\item{SimulationType MonteCarlo}\\
Starts the Monte Carlo part of RASPA. The particular ensemble is not specified but implicitly deduced from
the specified Monte Carlo moves. Note that a MD-move can be used for hybrid MC/MD.
\item{SimulationType MolecularDynamics}\\
Starts the Molecular Dynamics part of RASPA. The ensemble is explicitly specified.
\item{SimulationType Spectra}\\
Starts the computation of the vibrational analysis. Possible options include infra red spectrum at zero Kelvin,
powder diffraction, and mode analysis.
\item{SimulationType Minimization}\\
Starts the minimization routine. It produces configurations and crystal structures at zero Kelvin.
\item{SimulationType Visualization}\\
Output VTK-files for snapshots and crystal structures, including energy surface pictures.
\item{SimulationType BarrierCrossing}\\
Routine for the dynamical correction of dynamically corrected Transition State Theory.
\item{SimulationType Numerical}\\
Computes all the forces numerically from the energy and compares them to the analytical expressions.
Also the strain-derivative tensor (related to the stress tensor), and the second derivative of
the energy with respect to strain, as well as the Hessian matrix can be checked.
\item{SimulationType MakeGrid}\\
Creates pre-tabulated energy-grids for use in rigid frameworks.
\end{itemize}

\subsection*{Simulation duration}
\begin{itemize}
\item{NumberOfCycles [int]}\\
The number of cycles for the production run. 
For Monte Carlo a cycle consists of $N$ steps, where $N$ is the amount of
molecules with a minimum of 20 steps. This means that on average during each cycle on each molecule a
Monte Carlo move has been attempted (either successful or unsuccessful). For MD the number of cycles
is simply the amount of integration steps.
\item{NumberOfInitializationCycles [int]}\\
The number of cycles used to initialize the system using Monte Carlo. This can be used for both Monte Carlo
as well as Molecular Dynamics to quickly equilibrate the positions of the atoms in the system.
\item{NumberOfEquilibrationCycles [int]}\\
For Molecular Dynamics  it is the number of MD steps to equilibrate the velocities in the systems. After this 
equilibration the production run is started. For Monte Carlo, in particular CFMC, the equilibration-phase is used
to measure the biasing factors.
\end{itemize}

\subsection*{Restart and crash-recovery}
\begin{itemize}
\item{RestartFile [yes$|$no]}\\
Reads the positions, velocities, and force from the directory `RestartInitial'. Any creation of molecules
in the `simulation.input' file will be in addition and after this first read from file. This is useful to
load initial positions of cations for example, and after that create adsorbates. The restart file
is written at `PrintEvery' intervals.
\item{ContinueAfterCrash [yes$|$no]}\\
Write a binary file containing the complete status of the program. 
The file name is `binary\_restart.dat' and is located in the directory `CrashRestart'.
With this option to `yes' the presence of this file will result in continuation from
the point where the program was at the moment of outputting this file.
The file can be quite big (several hundreds of megabytes) and will be outputted every
`WriteBinaryRestartFileEvery' cycles.
\item{WriteBinaryRestartFileEvery [int]}\\
The output frequency (i.e. every [int] cycles) of the crash-recovery file.
\end{itemize}

\subsection*{Printing options}
\begin{itemize}
\item{PrintEvery [int]}\\
Prints the loadings (when a framework is present) and energies every [int] cycles. For MD information
like energy conservation and stress are printed.
\item{PrintPropertiesEvery [int]}\\
Output running averages of many properties (i.e. Henry coefficients and elastic constants).
\item{PrintForcefieldToOutput [yes$|$no]}\\
Prints the force field information to the output-file.
Default: yes.
\item{PrintPseudoAtomsToOutput [yes$|$no]}\\
Prints the pseudo-atom information to the output-file.
Default: yes.
\item{PrintMoleculeDefinitionToOutput [yes$|$no]}\\
Prints the molecule definition information to the output-file.
Default: yes.
\end{itemize}

\subsection*{Force field definitions}
\begin{itemize}
\item{ChargeFromChargeEquilibration [yes$|$no]}\\
Compute the charges of the framework using the `charge-equilibration'-method.
\item{SymmetrizeFrameworkCharges [yes$|$no]}\\
All charges of the framework are made equivalent for equivalent framework atoms. 
Using regular charge-equilibration the charges are different for symmtrically equivalent
framework atoms, and this options restores the symmetry.
\item{ForceField [string]}\\
Reads in the force field [string], first the file `pseudo\_atoms.def' is read, then 
`force\_field\_mixing\_rules.def' and finally `force\_field.def'. The latter overwrites
general settings for interactions based on mixing rules with specific ones for individual
interactions. 

Note that if any of these files are in the working directory then these will read and used instead of the
ones in `\$\{RASPA\_DIR\}/simulations/share/raspa/forcefield/[string]'.

\item{CutOffVDW [real]}\\
The cutoff of the Van der Waals potentials. Interactions longer then this distance are omitted from the
energy and force computations. The potential can either be shifted to zero at the cutoff, or interactions
can just neglected after the cut off, or the remainder of the potential energy can be approximated using
tail corrections. This is specified in the force field files and can be specified globally or
for each interaction individually.
\item{CutOffVDWSwitch [real]}\\
The distance at which VDW switching will start. The smoothing will make sure the value and derivatives are zero at the cutoff.
The default: 0.9 times the CutOff.
\item{CutOffChargeCharge [real]}\\
The cutoff of the charge-charge potential. The potential is truncated at the cutoff and only shifted when `ChargeMethod CoulombShifted'
or `ChargeMethod CoulombSmoothed' is used.
No tail-corrections are (or can be) applied. The only way to include the long-range part is to use `ChargeMethod Ewald'.
The parameter is also used in combination with the
Ewald precision to compute the number of wave vectors and Ewald parameter $\alpha$.
For the Ewald summation using rather large unit cells, a charge-charge cutoff of about half the smallest box-length would be advisable
in order to avoid the use of an excessive amount of wave-vectors in Fourier space. For non-Ewald methods the cutoff should be as large
as possible (greater than about 30 \AA).
\item{CutOffChargeChargeSwitch [real]}\\
The distance at which charge-charge switching will start. The smoothing will make sure the value and derivatives are zero at the cutoff.
The default: 0.65 times the CutOff.
\item{CutOffChargeBondDipole [real]}\\
The cutoff of the charge-bonddipole potential.
\item{CutOffChargeBondDipoleSwitch [real]}\\
The distance at which charge-bonddipole switching will start. The smoothing will make sure the value and derivatives are zero at the cutoff.
The default: 0.70 times the CutOff.
\item{CutOffBondDipoleBondDipole [real]}\\
The cutoff of the bonddipole-bonddipole potential.
\item{CutOffBondDipoleBondDipoleSwitch [real]}\\
The distance at which bonddipole-bonddipole switching will start. The smoothing will make sure the value and derivatives are zero at the cutoff.
The default: 0.75 times the CutOff.


\item{OmitAdsorbateAdsorbateVDWInteractions [yes$|$no]}\\
Omits the Van der Waals interactions between adsorbates.
\item{OmitAdsorbateAdsorbateCoulombInteractions [yes$|$no]}\\
Omits the Coulombic (i.e. Ewald) interactions between adsorbates.
\item{OmitInterMolecularInteractions [yes$|$no]}\\
Omits the interactions between all molecules (only interactions with the framework). This also
works with the Ewald summation on. The options implies the setting of both
\begin{itemize}
  \item{OmitAdsorbateAdsorbateVDWInteractions [yes$|$no]}
  \item{OmitAdsorbateAdsorbateCoulombInteractions [yes$|$no]}
\end{itemize}
\item{InternalFrameworkLennardJonesInteractions [yes$|$no]}\\
Compute the Van der Waals interaction of the flexible framework. The Demontis flexible model for silicalite
is defined with only bond, bend, and torsion for example. One can use this option and also use
`Charge None'.
\item{$\begin{array}{l}\text{RemoveBondNeighboursFromLongRangeInteraction [yes$|$no]}\\
      \text{RemoveBendNeighboursFromLongRangeInteraction [yes$|$no]}\\
      \text{RemoveTorsionNeighboursFromLongRangeInteraction [yes$|$no]}\end{array}$}\\
After construction of the connectivity table all interactions are removed from Van der Waals and charge
interactions that are defined as 1-2 (i.e. bonds), 1-3 (i.e. bends, Urey-Bradley) and 1-4 (i.e. torsion, inversion-bend) respectively.
\item{Remove12NeighboursFromChargeChargeInteraction [yes$|$no]\\
      Remove13NeighboursFromChargeChargeInteraction [yes$|$no]\\
      Remove14NeighboursFromChargeChargeInteraction [yes$|$no]}\\
Remove all 1-2, 1-3, and/or 1-4 interactions within the framework from the long-range charge-charge
interaction within the flexible framework respectively.
\item{Remove12NeighboursFromChargeBondDipoleInteraction [yes$|$no]\\
      Remove13NeighboursFromChargeBondDipoleInteraction [yes$|$no]\\
      Remove14NeighboursFromChargeBondDipoleInteraction [yes$|$no]}\\
Remove all 1-2, 1-3, and/or 1-4 interactions within the framework from the long-range charge-bond dipole
interaction within the flexible framework respectively.
\item{Remove12NeighboursFromBondDipoleBondDipoleInteraction [yes$|$no]\\
      Remove13NeighboursFromBondDipoleBondDipoleInteraction [yes$|$no]\\
      Remove14NeighboursFromBondDipoleBondDipoleInteraction [yes$|$no]}\\
Remove all 1-2, 1-3, and/or 1-4 interactions within the framework from the long-range bond dipole-bond dipole
interaction within the flexible framework respectively.
\end{itemize}

\subsection*{Thermostat and barostat parameters}
\begin{itemize}
\item{ExternalTemperature [list-of-reals]}\\
The external temperature in Kelvin for each system. Because the system is in contact with this imaginary reservoir the
average temperature of the system can be controlled. Default: 298K.
\item{ExternalPressure [list-of-reals]}\\
The external pressure in Pascal for each system. Because the system is in contact with this imaginary reservoir the
average pressure of the system can be controlled. 
\item{ThermostatChainLength [int]}\\
The length of the chain to thermostat the system. Default: 5.
\item{BarostatChainLength [int]}\\
The length of the chain to thermostat the volume and/or cell parameters. Default 5.
\item{NumberOfYoshidaSuzukiSteps [int]}\\
The number of Yoshida/Suzuki multiple timesteps.
\item{TimeScaleParameterThermostat [real]}\\
The time scale on which the system thermostat evolves. Default: 0.15 ps.
\item{TimeScaleParameterBarostat [real]}\\
The time scale on which the thermostat for the volume and/or cell parameters evolve. Default: 0.15 ps.
\end{itemize}

\subsection*{Molecular dynamics parameters}
\begin{itemize}
\item{TimeStep [real]}\\
The time step in picoseconds for MD integration. Default value: 0.0005 ps (0.5 fs).
\item{Ensemble [list-of-{\scriptsize NVE$|$NVT$|$NPT$|$NPH$|$NPTPR$|$NPHPR}]}\\
Sets the ensemble as a list of NVE,NVT, NPT, NPH, NPTPR, or NPHPR for each system. If only a single ensemble is given, it is used for all systems.
The given ensemble will be used for both initialization as well as the production run.
\begin{itemize}
\item{NVE}\\
The micro canonical ensemble, the number of particle $N$, the volume $V$, and the energy $E$ are constant.
\item{NVT}\\
The canonical ensemble, the number of particle $N$, the volume $V$, and the average temperature $\left\langle P\right\rangle$
are constant. Instantaneous values for the temperature are fluctuating.
\item{NPT}\\
The isobaric-isothermal ensemble, the number of particle $N$, the average pressure $\left\langle P\right\rangle$, and the average temperature $\left\langle P\right\rangle$
are constant. Instantaneous values for the pressure and temperature are fluctuating.
\item{NPH}\\
The isoenthalpic-isobaric ensemble, the number of particle $N$, the average pressure $\left\langle P\right\rangle$, and the enthalpy $H$
are constant. Instantaneous values for the pressure and temperature are fluctuating.
\item{NPTPR}\\
The isobaric-isothermal ensemble with a fully flexible cell (Parrinello-Rahman).
\end{itemize}
\item{InitEnsemble [list-of-{\scriptsize NVE$|$NVT$|$NPT$|$NPH$|$NPTPR$|$NPHPR}]}\\
Sets the ensemble as a list of NVE,NVT, NPH, NPTPR, or NPHPR for each system. If only a single ensemble is given, it is used for all systems.
The given ensemble will be only used for the initialization run.
\item{RunEnsemble [list-of-{\scriptsize NVE$|$NVT$|$NPT$|$NPH$|$NPTPR$|$NPHPR}]}\\
Set the ensemble as a list of NVE,NVT, NPH, NPTPR, or NPHPR for each system. If only a single ensemble is given, it is used for all systems.
The given ensemble will be only used for the production run.
\item{NPTPRCellType [list-of-{\scriptsize Regular$|$Monoclinic$|$RegularUpperTriangle$|$MonoclinicUpperTriangle$|$Isotropic$|$Anisotropic}]}\\
The type of constraints on the cell-matrix $\mathbf{h}$. Default: RegularUpperTriangle.
\begin{itemize}
\item{Regular}\\
If the pressure tensor is asymmetric ($P_{\alpha\beta}\not=P_{\beta\alpha}$) at a given instant of
time, then there will be a net torque acting on the cell that will cause it to rotate. Cell
rotations can be eliminated by using the symmetrized tensor $P_{\alpha\beta}= (P_{\alpha\beta}+P_{\beta\alpha})/2$ in the
equations of motion and setting the initial total angular momentum of the cell to zero.
This approach is formally implemented by constraining the force on the cell $\mathbf{g}=\mathbf{g}^T$.
All three angles $\alpha,\beta,\gamma$ are allowed to change, as well as the box lengths $\mathbf{a},\mathbf{b},\mathbf{c}$.
\item{Monoclinic}\\
All three box lengths $\mathbf{a},\mathbf{b},\mathbf{c}$ are allowed to vary, as well as cell angle $\beta$, but $\alpha=\gamma=90^\circ$.
\item{RegularUpperTriangle}\\
Only the upper triangular part of the cell matrix is used to eliminate rotation of the box.
All three angles $\alpha,\beta,\gamma$ are allowed to change, as well as the box lengths $\mathbf{a},\mathbf{b},\mathbf{c}$.
\item{MonoclinicUpperTriangle}\\
Only the upper triangular part of the cell matrix is used to eliminate rotation of the box.
All three box lengths $\mathbf{a},\mathbf{b},\mathbf{c}$ are allowed to vary, as well as cell angle $\beta$, but $\alpha=\gamma=90^\circ$.
\item{Isotropic}\\
All three box lengths $\mathbf{a}=\mathbf{b}=\mathbf{c}$ are allowed to vary isotropically, and the angles remain fixed $\alpha=\beta=\gamma=90^\circ$.
\item{Anisotropic}\\
All three box lengths $\mathbf{a},\mathbf{b},\mathbf{c}$ are allowed to vary \emph{independently}, but the angles remain fixed $\alpha=\beta=\gamma=90^\circ$.
\end{itemize}

\end{itemize}

\subsection*{Box parameters}
\begin{itemize}
\item{$\begin{array}{l}\text{Box [int]}\\ \text{[real] [real] [real]}\end{array}$}\\
Set the system [int] to type `Box' (other option is `Framework' when a framework is present).
The cell dimensions of rectangular box of system [int] in Angstroms. Default: 25 25 25 \AA.
\item{$\begin{array}{l}\text{BoxAngles [int]}\\ \text{[real] [real] [real]}\end{array}$}\\
Set the system [int] to type `Box' (other option is `Framework' when a framework is present).
The cell angles of rectangular box of system [int] in Angstroms. Default: 90$^\circ$ 90$^\circ$ 90$^\circ$.
\item{$\begin{array}{l}\text{BoxMatrix [int]}\\ 
      \text{[real] [real] [real]}\\
      \text{[real] [real] [real]}\\
      \text{[real] [real] [real]}\end{array}$\\
       }\\
Set the system [int] to type `Box' (other option is `Framework' when a framework is present).
The $3\times3$ cell matrix of system [int], given as three vectors (as columns). 
This is the most general form and any box can be specified in this way. Units of the vectors are Angstrom.
\end{itemize}


\subsection*{Framework parameters}
\begin{itemize}
\item{Framework [int]}\\
Set the system [int] to type `Framework' (other option is `Box' when no framework is present).
All other options listed in the section framework parameters refer to this system, so make sure this is before any other framework options.
\item{FrameworkName [string]}\\
Loads the framework with name [string]. Several frameworks can be read per system, 
which is useful for to study interpenetration of frameworks. Here the frameworks
are allowed to move independently from each other.
\item{HeliumVoidFraction [real]}\\
The void fraction as measure by probing the structure with helium a room temperature. This quantity has to be obtained from a separate
simulation and is essential to compute the \emph{excess}-adsorption during the simulation.
\item{UnitCells [int] [int] [int]}\\
The number of unit cells in x,y, and z direction for the system. The full cell will contain the unit cells, and periodic boundary conditions
will be applied on the box level (\emph{not} on a unit cell level).
\item{ShiftUnitCells [real] [real] [real]}\\
Shift the fractional positions so that the center of a framework can be altered.
\item{FlexibleFramework [yes$|$no]}\\
Allow the current framework of the current system to be fully flexible. The name of the flexible model is provided using the
`FrameworkDefinitions [string]' input option.
\item{FrameworkDefinitions [string]}\\
The force field name [string] of the flexible framework. The file is read even when `FlexibleFramework no' is specified (the reason is that 
framework bond-dipoles are defined using the `framework.def' file).
\item{ModifyFrameworkAtomConnectedTo [atom-type-1] [atom-type-2] [atom-type-3] [atom-type-4]}\\
Modifies the atom-type-1 to atom-type-2, always if atom-type-3 and atom-type-4 are omitted, or only it is connected to atom-type-3 when atom-type-3 is specified,
or only when it is connected to both atom-type-3 and atom-type-4 if both are specified.
\item{ModifyFrameworkDimer [atom-type-1] [atom-type-2] [atom-type-3] [atom-type-4]}\\
Modifies the connected atom-type-1 and atom-type-2 dimer to atom-type-3 and atom-type-4.
\item{ModifyFrameworkTriple [atom-type-1] [atom-type-2] [atom-type-3] [atom-type-4] [atom-type-5] [atom-type-6]}\\
Modifies the connected triple atom-type-1,atom-type-2,atom-type-3 to atom-type-4,atom-type-5,atom-type-6.
\item{RemoveAtomNumberCodeFromLabel [yes$|$no]}\\
Reading structure-files: the number is removed from the framework atom-types, e.g.\ \verb=`O1'=, \verb=`O2'=, \verb=`O3'=, etc.\ are mapped to \verb=`O'=.
\item{AddAtomNumberCodeToLabel [yes$|$no]}\\
Writing structure-files: the number is added to the framework atom-types, e.g.\ \verb=`O'= are mapped to \verb=`O1'=, \verb=`O2'=, \verb=`O3'=, etc.
\item{RestrictFrameworkAtomsToBox [yes$|$no]}\\
Restricts (places back) atoms to the unit cell dimensions, i.e. fractional positions between 0 and 1.
\item{ReadCIFAsCartesian [yes$|$no]}\\
Reads the position listed in the CIF-file as Cartesian. Only applicable to P1 systems (no symmetry).
\end{itemize}

\subsection*{System moves}

\begin{itemize}
\item{FrameworkChangeMoveProbability [real]}\\
The probability per cycle to randomly translate a framework atom. During this move the number of inner cycles is the amount of framework atoms, with a maximum of 500.
This move is applicable to relatively rigid structures like zeolites. For other structure where movement is caused by collective behavior (for example, the rotation of a phenyl-ring
in a metal-organic framework) the MC/MD move is more convenient. Such movement is hardly sampled at all by individual MC translation moves.
\item{VolumeChangeProbability [real]}\\
The probability per cycle to attempt a volume-change. Rigid molecules are scaled by center-of-mass, while flexible molecules and the framework is atomically scaled.
\item{VolumeChangeDirection [A$|$B$|$C$|$AB$|$AC$|$BC$|$ABC]}\\
Change the volume of the unit cell along a particular direction/directions in Monte Carlo. Default: ABC.
\item{BoxShapeChangeProbability [real]}\\
The probability per cycle to attempt a shape-change of the box. One of the 6 upper triangular elements of the box matrix is randomly chosen.
Rigid molecules are scaled by center-of-mass, while flexible molecules and the framework is atomically scaled.
\item{GibbVolumeChangeProbability [real]}\\
The probability per cycle to attempt a Gibbs volume-change MC move during a Gibbs ensemble simulation. The total volume of the two boxes 
(usually one for the gas phase, one for the liquid phase) remains constant, but the individual volume of the boxes are changed.
The volumes are changed by a random change in $\ln(V_I/V_{II})$.
\item{HybridNVEMoveProbability [real]}\\
The probability per cycle to attempt a hybrid Monte Carlo move using Molecular Dynamics in the NVE-ensemble.
The whole system is integrated using Newton's equations of motion. The new configuration
is then accepted or rejected using the standard MC rule. Note that the difference in energy $\Delta U$ is the integration error. The integration time step is set using
`TimeStep'.
\item{NumberOfHybridNVESteps [int]}\\
The number of integration steps for the hybrid MC/MD NVE move. Default: 5.
\item{ParallelTemperingProbability [real]}\\
A move where two neighboring systems 
are swapped that differ in their temperature.
\item{HyperParallelTemperingProbability [real]}\\
A move where two neighboring systems 
are swapped that differ in their temperature
and chemical potentials.
\item{ParallelMolFractionProbability [real]}\\
A move where two neighboring systems (similar to parallel tempering)
are swapped that differ in their mol-fraction of components $A$ and $B$.
\item{ParallelMolFractionComponentA [int]}\\
The identifier of the first component.
\item{ParallelMolFractionComponentB [int]}\\
The identifier of the second component.
\item{ChiralInversionProbability [real]}\\
A move specifically designed for systems with chiral molecules
to change all $S$-molecules into $R$-molecules and vica versa.
Note that the spacegroup needs to be set. If you have a framework that is P1 but has higher symmetry then use \verb=`CalculateSpaceGroup yes'= to
determine the true space group of the framework. An error will be given if this move is impossible for your system (e.g.\ when the framework is chiral).
\end{itemize}

\subsection*{Component information}
\begin{itemize}
\item{Component [int] MoleculeName [string]}\\
Reads in the definition of component [int] using the file `\emph{molecule-name-string}.def' from the 
directory `\$\{RASPA\_DIR\}/share/raspa/molecules/\emph{molecule-definitions-string}'.
\item{MoleculeDefinitions [string]}\\
The type of the molecule. For example, there could an OPLS version of the molecule, or a TraPPE version, etc. This \emph{molecule-definitions-string} is actually the directory name
under which the molecule file is found in `\$\{RASPA\_DIR\}/share/raspa/molecules/'.
\item{StartingBead [int]}\\
The staring bead for the configurational bias Monte Carlo (CBMC). In CBMC the molecule is grown bead by bead biasing the growth towards energetically favorable configurations.
Certain operations, like the rotation MC move and Widom particle insertion, use this bead as the center of rotation and position of the probe molecule, respectively.
\item{BlockPockets [yes$|$no]}\\
Block certain pockets in the simulation volume. The growth of a molecule is not allowed in a blocked pocket. A typical example is the sodalite cages in FAU and LTA-type zeolites,
these are not accessible to molecules like methane and bigger.
\item{BlockPocketsFileName [string]}\\
The file name for the definitions of all the blocking spheres.
\item{MolFraction [real]}\\
The mol fraction of this component in the mixture. The values can be specified relative to other components, as the fractions are normalized afterwards.
The partial pressures for each component are computed from the total pressure and the mol fraction per component.
\item{FugacityCoefficient [real]}\\
The fugacity coefficient for the current component. For values 0 (or by not specifying this line), the fugacity coefficients are automatically computed using the Peng-Robinson
equation of state. Note the critical pressure, critical temperature, and acentric factor need to be specified in the molecule file.
\item{Intra14VDWScalingValue [real]}\\
The scaling factor for intra-molecular 1-4 van der Waals interactions. For example: OPLS uses a factor of $\frac{1}{2}$.
\item{Intra14ChargeChargeScalingValue [real]}\\
The scaling factor for intra-molecular 1-4 charge/charge interactions. For example: OPLS uses a factor of $\frac{1}{2}$.
\item{IdealGasRosenbluthWeight [real]}\\
The ideal Rosenbluth weight is the growth factor of the CBMC algorithm for a single chain in an empty box. The value only depends on temperature and therefore needs to be computed
only once. For adsorption, specifying the value in advance is convenient because the applied pressure does not need to be corrected afterwards (the Rosenbluth weight corresponds to a shift
in the chemical potential reference value, and the chemical potential is directly obtained from the fugacity). For equimolar mixtures this is essential.
\item{GibbsSwapProbability [real]}\\
The relative probability to attempt a Gibbs swap MC move for the current component. The `GibbsSwapMove' transfers a randomly selected particle from one box to the other
(50\% probability to transfer a particle from box I to II, an 50\% visa versa).
\item{TranslationProbability [real]}\\
The relative probability to attempt a translation move for the current component. A random displacement is chosen in the allowed directions (see `TranslationDirection').
Note that the internal configuration of the molecule is unchanged by this move. The maximum displacement is scaled during the simulation to achieve an acceptance
ratio of 50\%.
\item{TranslationDirection 
      \begin{small}
       [X$|$Y$|$Z$|$XY$|$XZ$|$YZ$|$XYZ$|$A$|$B$|$C$|$AB$|$AC$|$BC$|$ABC$|$\\
      \hspace*{2cm}ORTHOGONAL\_TO\_AB\_DIR$|$ORTHOGONAL\_TO\_AC\_DIR$|$ORTHOGONAL\_TO\_BC\_DIR$|$\\
      \hspace*{2cm}ORTHOGONAL\_TO\_O\_AB\_DIR$|$ORTHOGONAL\_TO\_O\_AC\_DIR$|$ORTHOGONAL\_TO\_O\_BC\_DIR$|$\\
      \hspace*{2cm}ORTHOGONAL\_TO\_A\_BC\_DIR$|$ORTHOGONAL\_TO\_B\_AC\_DIR$|$ORTHOGONAL\_TO\_C\_AB\_DIR$|$\\
      \hspace*{2cm}ORTHOGONAL\_TO\_O\_ABC\_DIR]
     \end{small}}\\
Specifies the allowed translation direction for the current component. Useful to sampling configuration with the starting bead restricted to a plane, i.e. see dcTST.
Default: XYZ.
\item{RandomTranslationProbability [real]}\\
The relative probability to attempt a random translation move for the current component. The displacement is chosen such that any position in the box can reached. It is therefore
similar as reinsertion, but `reinsertion' changes the internal conformation of a molecule and uses biasing.
\item{RotationProbability [real]}\\ 
The relative probability to attempt a random rotation move for the current component. The rotation is around the starting bead. A random vector on a sphere
is generated, and the rotation is random around this vector.
\item{CBMCProbability [real]}\\
The relative probability to attempt a partial reinsertion move for the current component. Part of the molecule is regrown, while part of the molecule can remain fixed.
The list of partial reinsertion moves is specified in the `molecule.def' file.
\item{ReinsertionProbability [real]}\\
The relative probability to attempt a full reinsertion move for the current component. Multiple first beads are chosen, and one of these is selected according to its Boltzmann weight.
The remaining part of the molecule is grown using biasing. This move is very useful, and often necessary, to change the internal configuration of flexible molecules.
\item{SwapProbability [real]}\\
The relative probability to attempt a insertion or deletion move. Whether to insert or delete is decided randomly with a probability of 50\% for each.
The swap move imposes a chemical equilibrium between the system and an imaginary particle reservoir for the current component. The move starts with multiple first bead, and
grows the remainder of the molecule using biasing.
\item{WidomProbability [real]}\\
The relative probability to attempt a Widom particle insertion move for the current component. The Widom particle insertion moves measure the chemical potential
and can be directly related to Henry coefficients and heats of adsorption.
\item{SurfaceAreaProbability [real]}\\
The relative probability to attempt a surface-area move for the current component.
\item{ReinsertionInPlaceProbability [real]}\\
The relative probability to attempt a reinsertion-in-place move for the current component. The reinsertion position is the current position of the starting bead of the randomly selected
molecule. Alternatively, one can use the partial reinsertion move leaving one bead fixed. The move is very useful to sample configuration on a plane for dcTST to change
the internal configuration, e.g. bonds, bends, torsions, etc.
\item{IdentityChangeProbability [real]}\\
The relative probability to attempt an identity-change move for the current component. A molecule of type $A$ is reinsertion, in the same place as the starting bead of $A$, as type $B$ using
the starting bead of component $B$. The $A-B$ list is defined using `IdentityChangesList' defining $B$ for each component $A$, i.e. the current component can be reinserted into
any component defined in the `IdentityChangesList' list, and from that list the component is chosen randomly.
\item{NumberOfIdentityChanges [int]}\\
The number of `IdentityChangesList' elements for the current component.
\item{IdentityChangesList [list-of-int]}\\
The list of components that the current component can be changed into. The identity-change move will randomly choose the new component from this list.
\item{GibbsIdentityChangeProbability [real]}\\
The relative probability to attempt an identity change for the current component in the Gibbs ensemble.
It is a very useful move to for mixture of $n$ components. Out of the $n$ components, two components $i\not=j$ are selected at random.
At random, it is selected to switch the identity of component $i$
in box $I$ or in box $II$, and the identity of the component $j$ in the other box.
In each box, a particle is selected at random which matches the desired identity.
\item{NumberOfGibbsIdentityChanges [int]}\\
The number of `GibbsIdentityChangesList' elements for the current component.
\item{GibbsIdentityChangesList [list-of-int]}\\
The list of components that the current component can be changed into. The Gibbs-identity-change move will randomly choose the new component from this list.
\item{ExtraFrameworkMolecule [yes$|$no]}\\
There are two major types of molecules, `Adsorbates' and `Cations'. The `ExtraFrameworkMolecule' keyword sets whether the current component is a `Cation' (yes) or a `Adsorbate' (no).
Energies in the output as splitted in Host-Host, Host-Adsorbate, Host-Cation, Adsorbate-Adsorbate, Cation-Cation, and Adsorbate-Cation. The distinction in two types of molecule
is sometimes necessary. For example, consider a mixture of components, where polarization needs to be neglected between certain components (because they are parameterized without).
The water model `rpol' is defined including polarization, but CO2 using TraPPE is not. One can define water as `Adsorbate', CO2 as `Cation' and neglect polarization between cations.
\item{RestrictEnantionface [yes$|$no]}\\
Restricts all MC-moves to the enantioface defined by `Enantioface'. Moves that result in an opposite enantioface are rejected.
\item{Enantioface [Re$|$Si]}\\
The enantioface of the component, either `Re' or `Si'.
\item{EnantiofaceAtoms \begin{footnotesize}[F$|$A$|$C] [int] [int] [F$|$A$|$C] [int] [int] [F$|$A$|$C] [int] [int]
 [F$|$A$|$C] [int] [int] [F$|$A$|$C] [int] [int]\end{footnotesize}}\\
The definition of the enantioface based on 5 atoms. The first 4 form a torsion, as well as the first 3 and the last atom. These two torsions form the definition
of the enantioface.
\item{CreateNumberOfMolecules [int]}\\
The number of molecule to create for the current component. Note these molecules are \emph{in addition} to anything read in by using a restart-file. Usually, when the restart-file
is used the amount here should be put back to zero. A warning, putting this value unreasonably high results in an infinite loop. The routine accepts molecules that are grown causing
no overlap (energy smaller than `EnergyOverlapCriteria'). Also the initial starting configurations are far from optimal and substantial equilibration is needed to reduce the energy.
However, the CBMC growth is able to reach very high densities.
\end{itemize}


\subsection*{Options to measure properties}
\begin{itemize}

\item{ComputeNumberOfMoleculesHistogram [yes$|$no]}\\
Sets whether or not to compute the histograms of the number of molecules for the current system.
In open ensembles the number of molecules fluctuates.
\begin{itemize}
\item{WriteNumberOfMoleculesHistogramEvery [int]}\\
Output the histogram every [int] cycles.
\item{NumberOfMoleculesRange [real]}\\
The range of the histograms.
\item{NumberOfMoleculesHistogramSize [int]}\\
The number of elements of the histograms.
\end{itemize}

\item{ComputeDistanceHistograms [yes$|$no]}\\
Sets whether or not to compute the histograms of specified distance pairs for the current system.
A directory `DistanceHistograms' is created containing the histograms for each system.
\begin{itemize}
\item{WriteDistanceHistogramEvery [int]}\\
Output the distance histograms every [int] cycles.
\item{MaxRangeDistanceHistogram [real]}\\
The range of the histograms.
\item{NumberOfElementsDistanceHistogram [int]}\\
The number of elements of the histograms.
\item{DistanceHistogramDefinition [F$|$A$|$C] [int] [int] [F$|$A$|$C] [int] [int]}\\
Define a distance histogram between two atoms.
\end{itemize}

\item{ComputeBendAngleHistograms [yes$|$no]}\\
Sets whether or not to compute the bend-angle histograms of specified trimers of atoms for the current system.
A directory `BendAngleHistograms' is created containing the histograms for each system.
\begin{itemize}
\item{WriteBendAngleHistogramEvery [int]}\\
Output the distance histograms every [int] cycles.
\item{MaxRangeBendAngleHistogram [real]}\\
\item{NumberOfElementsBendAngleHistogram [int]}\\
\item{BendAngleHistogramDefinition [F$|$A$|$C] [int] [int] [F$|$A$|$C] [int] [int] [F$|$A$|$C] [int] [int]}\\
\end{itemize}

\item{ComputeDihedralAngleHistograms [yes$|$no]}\\
Sets whether or not to compute the dihedral-angle histograms of specified quads of atoms for the current system.
A directory `DihedralAngleHistograms' is created containing the histograms for each system.
\begin{itemize}
\item{WriteDihedralAngleHistogramEvery [int]}\\
Output the distance histograms every [int] cycles.
\item{MaxRangeDihedralAngleHistogram [real]}\\
\item{NumberOfElementsDihedralAngleHistogram [int]}\\
\item{DihedralAngleHistogramDefinition \small{[F$|$A$|$C] [int] [int] [F$|$A$|$C] [int] [int] [F$|$A$|$C] [int] [int] [F$|$A$|$C] [int] [int]}}\\
\end{itemize}

\item{ComputeAngleBetweenPlanesHistograms [yes$|$no]}\\
Sets whether or not to compute the histograms of angles between specified planes for the current system.
A directory `AngleBetweenPlanesHistograms' is created containing the histograms for each system.
\begin{itemize}
\item{WriteAngleBetweenPlanesHistogramEvery [int]}\\
Output the distance histograms every [int] cycles.
\item{MaxRangeAngleBetweenPlanesHistogram [real]}\\
\item{NumberOfElementsAngleBetweenPlanesHistogram [int]}\\
\item{AngleBetweenPlanesHistogramDefinition [F$|$A$|$C] [int] [int] [F$|$A$|$C] [int] [int] [F$|$A$|$C] [int] [int]
        [F$|$A$|$C] [int] [int] [F$|$A$|$C] [int] [int] [F$|$A$|$C] [int] [int]}\\
\end{itemize}


\item{ComputePSD [yes$|$no]}\\
Sets whether or not to compute the pore-size distribution (PSD) for the current system.
A directory `PoreSizeDistributionHistogram' is created containing the output 
`HistogramPoreSizeDistribution.dat' per system.
\begin{itemize}
\item{WritePSDEvery [int]}\\
Output the PSD every [int] cycles.
\item{PSDProbeDistance [Minimum$|$Sigma]}\\
Sets whether to use the minimum of the potential $\sigma^{1/6}$ as the probe distance or whether to use $\sigma$.
\item{HistogramSizePoreSizeDistribution [int]}\\
default: 100.
\item{MaxRangePoreSizeDistribution [real]}\\
default: 10.
\end{itemize}


\item{ComputeRDF [yes$|$no]}\\
Sets whether or not to compute the radial distribution function (RDF) for the current system.
A directory `RadialDistributionFunctions' is created containing the output per system.
The RDF is computed for each atom type pair unless the option `print' flag in `pseudo\_atoms.def'
is `no'.
\begin{itemize}
\item{WriteRDFEvery [int]}\\
Output the RDF every [int] cycles.
\end{itemize}

\item{ComputeMSD [yes$|$no]}\\
Sets whether or not to compute the mean-squared displacement (MSD) for the current system using a modified order-N algorithm.
A directory `MSDOrderN' is created containing the output per system. The output consists of files containing self-msd data per
component, the total self-msd, the Onsager msd for each component pair, and the the total Onsager msd.
The units in the files are \AA$^2$ for the msd, and ps for time.
  \begin{itemize}
    \item{WriteMSDEvery [int]}\\
     Output the MSD every [int] cycles.
    \item{SampleMSDEvery [int]}\\
    Samples every [int] integration steps. Default: 1.
    \item{ComputeIndividualMSD [yes$|$no]}\\
    Computes the msd, not only per component, but also per molecule.
    \item{NumberOfBlocksMSD [int]}\\
    The  number of blocks for the order-$n$ correlation measurement. Each block represent a different time-scale of sampling. Default: 25.
    \item{NumberOfBlockElementsMSD [int]}\\
    The number of elements in each block. For example, if the number is 10, then the first block samples: $1,2,3,\dots,10$, the second block
    $10,20,30,\dots,100$, the third block $100,200,300,\dots,1000$, etc. Default: 25.
   \end{itemize}

\item{ComputeVACF [yes$|$no]}\\
Sets whether or not to compute the velocity autocorrelation function (VACF) for the current system using a modified order-N algorithm.
A directory `VACFOrderN' is created containing the output per system. The output consists of files containing self-vacf data per
component, the total self-vacf, the Onsager vacf for each component pair, and the the total Onsager vacf.
The files start with the integration diffusivity-values, computed using a generalization of the Simpson's rule 
(in the sense that it is exact for cubic polynomials and is valid for an odd as well as even number of intervals).
The units in the files are \AA$^2$/ps for velocity, and ps for time.
  \begin{itemize}
    \item{WriteVACFEvery [int]}\\
     Output the VACF every [int] cycles.
    \item{SampleVACFEvery [int]}\\
    Samples every [int] integration steps. Default: 5.
    \item{ComputeIndividualVACF [yes$|$no]}\\
    Computes the vacf, not only per component, but also per molecule.
    \item{NumberOfBlocksVACF [int]}\\
    The  number of blocks for the order-$n$ correlation measurement. Each block represent a different time-scale of sampling. Default: 10.
    \item{NumberOfBlockElementsVACF [int]}\\
    The number of elements in each block. For example, if the number is 10, then the first block samples: $1,2,3,\dots,10$, the second block
    $10,20,30,\dots,100$, the third block $100,200,300,\dots,1000$, etc. Default: 5000.
   \end{itemize}

\item{ComputeRVACF [yes$|$no]}\\
Sets whether or not to compute the rotational velocity autocorrelation function 
(RVACF) for the current system using a modified order-N algorithm.
A directory `RVACFOrderN' is created containing the output per system. The output consists of files containing self-rvacf data per
component, the total self-rvacf, the Onsager rvacf for each component pair, and the the total Onsager rvacf.
The files start with the integration diffusivity-values, computed using a generalization of the Simpson's rule
(in the sense that it is exact for cubic polynomials and is valid for an odd as well as even number of intervals).
The units in the files are \AA$^2$/ps for velocity, and ps for time.
  \begin{itemize}
    \item{WriteRVACFEvery [int]}\\
     Output the RVACF every [int] cycles.
    \item{SampleRVACFEvery [int]}\\
    Samples every [int] integration steps. Default: 5.
    \item{ComputeIndividualRVACF [yes$|$no]}\\
    Computes the vacf, not only per component, but also per molecule.
    \item{NumberOfBlocksRVACF [int]}\\
    The  number of blocks for the order-$n$ correlation measurement. Each block represent a different time-scale of sampling. Default: 10.
    \item{NumberOfBlockElementsRVACF [int]}\\
    The number of elements in each block. For example, if the number is 10, then the first block samples: $1,2,3,\dots,10$, the second block
    $10,20,30,\dots,100$, the third block $100,200,300,\dots,1000$, etc. Default: 5000.
   \end{itemize}

\item{ComputeMOACF [yes$|$no]}\\
Sets whether or not to compute the molecular orientation velocity autocorrelation function
(MOACF) for the current system using a modified order-N algorithm.
A directory `MOACFOrderN' is created containing the output per system. The output consists of files containing self-moacf data per
component and the total self-rvacf.
The units in the files are rad$^2$/ps for velocity, and ps for time.
  \begin{itemize}
    \item{WriteMOACFEvery [int]}\\
     Output the MOACF every [int] cycles.
    \item{SampleMOACFEvery [int]}\\
    Samples every [int] integration steps. Default: 5.
    \item{ComputeIndividualMOACF [yes$|$no]}\\
    Computes the moacf, not only per component, but also per molecule.
    \item{NumberOfBlocksMOACF [int]}\\
    The  number of blocks for the order-$n$ correlation measurement. Each block represent a different time-scale of sampling. Default: 10.
    \item{NumberOfBlockElementsMOACF [int]}\\
    The number of elements in each block. For example, if the number is 10, then the first block samples: $1,2,3,\dots,10$, the second block
    $10,20,30,\dots,100$, the third block $100,200,300,\dots,1000$, etc. Default: 5000.
   \end{itemize}


\item{ComputeMSDConventional [yes$|$no]}\\
Sets whether or not to compute the mean-squared displacement (MSD) for the current system using the conventional algorithm.
A directory `MSD' is created containing the output per system. The routine is available for legacy reasons,
the same results can be obtained using the order-N method and 1 block of size `BufferLengthMSD'.
The units in the files are \AA$^2$ for the msd, and ps for time.
  \begin{itemize}
    \item{WriteMSDConventionalEvery [int]}\\
     Output the MSD every [int] cycles. Default: 5000.
    \item{SampleMSDConventionalEvery [int]}\\
    Samples every [int] integration steps. Default: 1.
    \item{NumberOfBuffersMSDConventional [int]}\\
    The number of (overlapping) buffers with a different offset in time. Default: 20.
    \item{BufferLengthMSDConventional [int]}\\
    The length of the buffers. Default: 5000.
   \end{itemize}


\item{ComputeVACFConventional [yes$|$no]}\\
Sets whether or not to compute the velocity autocorrelation function (VACF) for the current system using the conventional algorithm.
A directory `VACF' is created containing the output per system. The routine is available for legacy reasons, 
the same results can be obtained using the order-N method and 1 block of size `BufferLengthVACF'.
The units in the files are \AA$^2$/ps for velocity, and ps for time.
  \begin{itemize}
    \item{WriteVACFConventionalEvery [int]}\\
     Output the VACF every [int] cycles. Default: 5000.
    \item{SampleVACFConventionalEvery [int]}\\
    Samples every [int] integration steps. Default: 1.
    \item{NumberOfBuffersVACFConventional [int]}\\
    The number of (overlapping) buffers with a different offset in time. Default: 20.
    \item{BufferLengthVACFConventional [int]}\\
    The length of the buffers. Default: 5000.
   \end{itemize}

\item{ComputeDensityHistograms [yes$|$no]}\\
Sets whether or not to compute a density histogram for the current system. For example, during adsorption it keeps track of the amount of molecules.

\item{ComputeEnergyHistogram [yes$|$no]}\\
Sets whether or not to compute a histogram of the energy for the current system. 
For example, during adsorption it keeps track of the total energy, the VDW energy,
the Coulombic energy, and the polarization energy. Output is written to  the directory
`EnergyHistograms'.
\begin{itemize}
\item{WriteEnergyHistogramEvery [int]}\\
Sets to print the energy histogram of the system every [int] cycles.
\item{EnergyHistogramSize [int]}\\
Sets the number of elements of the histogram. Default: 1000.
\item{EnergyHistogramLowerLimit [real]}\\
Sets the lower limit of the histogram. Default: -10000.
\item{EnergyHistogramUpperLimit [real]}\\
Sets the upper limit of the histogram. Default: 0.
\end{itemize}

\item{ComputeThermoDynamicFactor [yes$|$no]}\\
Sets whether or not to compute the thermodynamic factors of the energy for the current system. 
The output is written to the directory `ThermoDynamicFactor'.
\begin{itemize}
\item{WriteThermoDynamicFactorEvery [int]}\\
Sets to print the thermodynamic factors every [int] cycles.
\end{itemize}

\item{ComputeEndToEndDistanceHistograms [yes$|$no]}\\
Sets whether or not to compute a histogram for end-to-end distances of molecules for the current system.

\item{ComputePrincipleMomentsOfInertia [yes$|$no]}\\
Sets whether or not to compute the average principle moments of inertia of molecules for the current system.

\item{ComputeSpectra [yes$|$no]}\\
Sets whether or not to compute the Infra-Red (IR) spectra of molecules for the current system.
\begin{itemize}
\item{WriteSpectraEvery [int]}\\
Sets to print the spectra of molecules every [int] cycles.
\end{itemize}
\item{ComputeMoleculeProperties [yes$|$no]}\\
Sets whether or not to compute properties of molecules like average bond-lengths, average bend-angles etc. for the current system.
\item{PrintMoleculePropertiesEvery [int]}\\
Sets to print the properties of molecules every [int] cycles.
\item{ComputeSurfaceArea [yes$|$no]}\\
Sets whether or not to compute the surface.
\begin{itemize}
\item{SurfaceAreaProbeAtom [string]}\\
\item{SurfaceAreaSamplingPointsPerSphere [int]}\\
Sets the number of points to sampling a sphere per iteration.
\item{SurfaceAreaProbeDistance [Minimum$|$Sigma]}\\
Sets whether to use the minimum of the potential $\sigma^{1/6}$ as the probe distance or whether to use $\sigma$.
\end{itemize}
\item{DensityProfile [yes$|$no]}\\
\item{DensityProfileGridPoints [int] [int] [int]}\\
\item{ComputeElasticConstants [yes$|$no]}\\
Sets whether to compute elastic constants.
\item{ComputePowderDiffractionPattern [yes$|$no]}\\
Sets whether to compute the powder diffraction pattern for the framework.
\begin{itemize}
\item{DiffractionType [Xray$|$Neutron$|$Electron]}\\
Sets the diffraction type as xray-scattering, neutron-scattering, or electron-scattering, respectively.
\item{DiffractionRadiationType [chromium$|$iron$|$copper$|$molybdenum$|$silver$|$synchrotron]}\\
Sets the type of the diffraction radiation as chromium, iron, copper, molybdenum, silver, or synchrotron, respectively.
\item{WaveLengthType [Single$|$Double]}\\
Set the type of the beam as single or as a doublet.
\item{PeakShape [Gaussian$|$Lorentzian$|$PseudoVoigt]}\\
Sets the shape of the peaks as Gaussian, Lorentzian, or Pseudo-Voigt, respectively.
\item{WaveLength [real]}\\
Sets the wavelength of the diffraction beam.
\item{TwoThetaMin [real]}\\
Sets the minimum value of $2\theta$.
\item{TwoThetaMax [real]}\\
Sets the maximum value of $2\theta$.
\item{TwoThetaStep [real]}\\
Sets the step size of $2\theta$.
\item{PeakWidthModifierU [real]}\\
\item{PeakWidthModifierV [real]}\\
\item{PeakWidthModifierW [real]}\\
\end{itemize}
\item{ComputerNormalModes [yes$|$no]}\\
Sets whether to compute normal modes.
\begin{itemize}
\item{MinimumMode [int]}\\
Sets the minimum normal to compute.
\item{MaximumMode [int]}\\
Sets the maximum normal to compute.
\item{ModeResolution [int]}\\
\end{itemize}
\end{itemize}

\subsection*{Energy/force grid options}
\begin{itemize}
\item{UseTabularGrid [yes$|$no]}\\
Use a pre-tabulated grid for the energy and forces. Default: no.
\item{SpacingVDWGrid [real]}\\
The grid spacing of the Van der Waals potentials. Default: 0.15 Angstrom.
\item{SpacingCoulombGrid [real]}\\
The grid spacing of the Coulomb potential. Default: 0.15 Angstrom.
\item{GridTypes [list-of-strings]}\\
A list of atom-types for each of the used grids.
\end{itemize}

\subsection*{Minimization/Saddle point search}
\begin{itemize}
\item{MinimizationMethod [Baker]}\\
The Baker minimization method uses the eigenvalues/vectors to find a true minimum where all eigenvalues are positive. Newton-Raphson
uses the first and second derivatives, but not the eigenvalues/vectors.
The saddle point search can best be started from a minimum energy configuration. The algorithm walks up hill along the softest eigen mode
to find a first order saddle point.
\item{MinimizationVariables [Cartesian$|$Fractional]}\\
Whether the minimization is performed in Cartesian or fractional positions. For some crystal minimizations it might be more convenient
to choose fractional positions. An example is when one wants to keep a particular fractional position fixed during the minimization.
\item{MaximumNumberOfMinimizationSteps [int]}\\
The maximum number of minimization steps after which the minimization is stopped. Default: 10000.
\item{RMSGradientTolerance [real]}\\
Stopping criteria: the maximum allowed RMS gradient. Default: $10^{-6}$.
\item{MaxGradientTolerance [real]}\\
Stopping criteria: the maximum allowed gradient for each and every atom (and the strain elements for cell minimizations). Default: $10^{-6}$.
\item{MaximumStepLength [real]}\\
The maximum length of a minimization step. The length is dependent on the problem at hand.
A too low value converges slowly (i.e. the minimization takes more steps),
while a too high value might not converge at all. Default value: 0.3.
\item{FrameworkFixedInitialization [free$|$fixed]}\\
Sets all framework atoms as `free' or `fixed'. This command must preceed individual overwrites
and applies to the current system.
\item{AdsorbateFixedInitialization [free$|$fixed]}\\
Sets all adsorbate groups and atoms as `free' or `fixed'. This command must preceed individual overwrites
and applies to the current system.
\item{CationFixedInitialization [free$|$fixed]}\\
Sets all cation groups and atoms as `free' or `fixed'. This command must preceed individual overwrites
and applies to the current system.
\item{ActiveFrameworkAtom [int]}\\
Sets the atom of the current framework and system as `active'.
\item{ActiveFrameworkAtoms [int] [list-of-ints]}\\
Sets the [int] atoms listed in [list-of-ints] of the current framework and system as `active'.
\item{FixedFrameworkAtom [int]}\\
Sets the atom of the current framework and system as `fixed'.
\item{FixedFrameworkAtoms [int] [list-of-ints]}\\
Sets the [int] atoms listed in [list-of-ints] of the current framework and system as `fixed'.

\item{ActiveAdsorbateMolecule [int]}\\
Sets all atom and groups of the adsorbate molecule [int] as `active'. Applies to the current system.
\item{FixedAdsorbateMolecule [int]}\\
Sets all atom and groups of the adsorbate molecule [int] as `fixed'. Applies to the current system.
\item{ActiveAdsorbateAtom [int] [int]}\\
Sets an atom (second argument) of an adsorbate molecule (first argument) as `active'. Applies to the current system.
\item{FixedAdsorbateAtom [int] [int]}\\
Sets an atom (second argument) of an adsorbate molecule (first argument) as `fixed'. Applies to the current system.
\item{ActiveAdsorbateGroup [int] [int]}\\
Sets a group (second argument) of an adsorbate molecule (first argument) as `active'. Applies to the current system
and both center of mass and the orientation are set as `active'.
\item{FixedAdsorbateGroup [int] [int]}\\
Sets a group (second argument) of an adsorbate molecule (first argument) as `fixed'. Applies to the current system
and both center of mass and the orientation are set as`fixed'.
\item{ActiveAdsorbateGroupCenterOfMass [int] [int]}\\
Sets a group (second argument) of an adsorbate molecule (first argument) as `active'. Applies to the current system
and only the center of mass is set as `active'.
\item{FixedAdsorbateGroupCenterOfMass [int] [int]}\\
Sets a group (second argument) of an adsorbate molecule (first argument) as `fixed'. Applies to the current system
and only the center of mass is set as `fixed'.
\item{ActiveAdsorbateGroupOrientation [int] [int]}\\
Sets a group (second argument) of an adsorbate molecule (first argument) as `active'. Applies to the current system
and only the orientation is set as `active'.
\item{FixedAdsorbateGroupOrientation [int] [int]}\\
Sets a group (second argument) of an adsorbate molecule (first argument) as `fixed'. Applies to the current system
and only the orientation is set as `fixed'.

\item{ActiveCationMolecule [int]}\\
Sets all atom and groups of the cation molecule [int] as `active'. Applies to the current system.
\item{FixedCationMolecule [int]}\\
Sets all atom and groups of the cation molecule [int] as `fixed'. Applies to the current system.
\item{ActiveCationAtom [int] [int]}\\
Sets an atom (second argument) of an cation molecule (first argument) as `active'. Applies to the current system.
\item{FixedCationAtom [int] [int]}\\
Sets an atom (second argument) of an cation molecule (first argument) as `fixed'. Applies to the current system.
\item{ActiveCationGroup [int] [int]}\\
Sets a group (second argument) of an cation molecule (first argument) as `active'. Applies to the current system
and both center of mass and the orientation are set as `active'.
\item{FixedCationGroup [int] [int]}\\
Sets a group (second argument) of an cation molecule (first argument) as `fixed'. Applies to the current system
and both center of mass and the orientation are set as `fixed'.
\item{ActiveCationGroupCenterOfMass [int] [int]}\\
Sets a group (second argument) of an cation molecule (first argument) as `active'. Applies to the current system
and only the center of mass is set as `active'.
\item{FixedCationGroupCenterOfMass [int] [int]}\\
Sets a group (second argument) of an cation molecule (first argument) as `fixed'. Applies to the current system
and only the center of mass is set as `fixed'.
\item{ActiveCationGroupOrientation [int] [int]}\\
Sets a group (second argument) of an cation molecule (first argument) as `active'. Applies to the current system
and only the orientation is set as `active'.
\item{FixedCationGroupOrientation [int] [int]}\\
Sets a group (second argument) of an cation molecule (first argument) as `fixed'. Applies to the current system
and only the orientation is set as `fixed'.




\item{$\begin{array}{l}\text{FixAtomType [string]}\\\text{FixAtomTypes [int] [list-of-strings]}\end{array}$}\\
The atom-types that are considered fixed during the minimization. All other atoms/groups will be optimized.
If the atom-type is contained in a rigid unit, the entire unit will be frozen.
\item{DistanceConstraint [F$|$A$|$C] [int] [int] [F$|$A$|$C] [int] [int] [real]}\\
Defines a `hard' distance constraint between two atoms and/or groups, and the distance.
\item{AngleConstraint [F$|$A$|$C] [int] [int] [F$|$A$|$C] [int] [int] [F$|$A$|$C] [int] [int] [real]}\\
Defines a `hard' angular constraint between three atoms and/or groups, and the constraint angle.
\item{DihedralConstraint [F$|$A$|$C] [int] [int] [F$|$A$|$C] [int] [int]
    [F$|$A$|$C] [int] [int] [F$|$A$|$C] [int] [int] [real]}\\
Defines a `hard' dihedral constraint between four atoms and/or groups, and constraint dihedral.
\item{HarmonicDistanceConstraint [F$|$A$|$C] [int] [int] [F$|$A$|$C] [int] [int] [real] [real]}\\
Defines a `hard' distance constraint between two atoms and/or groups, and the distance.
\item{HarmonicAngleConstraint \begin{small}[F$|$A$|$C] [int] [int] [F$|$A$|$C] [int] [int] 
  [F$|$A$|$C] [int] [int]  [real] [real]\end{small}}\\
Defines a `hard' angular constraint between three atoms and/or groups, and the constraint angle.
\item{HarmonicDihedralConstraint \begin{footnotesize}[F$|$A$|$C] [int] [int] [F$|$A$|$C] [int] [int]
 [F$|$A$|$C] [int] [int] [F$|$A$|$C] [int] [int] [real] [real]\end{footnotesize}}\\
Defines a `hard' dihedral constraint between four atoms and/or groups, and constraint dihedral.
\end{itemize}

\subsection*{Monte Carlo settings}
\begin{itemize}
\item{MinimumInnerCycles [int]}\\
The minimum number of inner cycles for each cycle. Default: 20.
\item{NumberOfTrialPositions [int]}\\
The number of trial positions during the growth of a molecule. Default: 10.
\item{NumberOfTrialPositionsForTheFirstBead [int]}\\
The number of trial positions for the first bead. Default: 10.
\item{NumberOfTrialPositionsTorsion [int]}\\
The number of trial positions for torsions over a single bond. Default: 100.
\item{NumberOfTrialMovesPerOpenBead [int]}\\
The number of trial moves per open bead during CBMC. Default: 200.
\item{TargetAccRatioSmallMCScheme [real]}\\
\item{TargetAccRatioTranslation [real]}\\
\item{EnergyOverlapCriteria [real]}\\
The energy criteria to consider an energy as `overlap'. Default: $10^5$ K.
\item{MinimumRosenbluthFactor [real]}\\
The minimum Rosenbluth weight, values lower are consider to be `overlapping'. Default: $10^{-150}$.
\end{itemize}

\subsection*{Biasing options}
\begin{itemize}
\item{BiasingDirection [A$|$B$|$C$|$AB\_DIAGONAL$|$AC\_DIAGONAL$|$BC\_DIAGONAL$|$\\
              \hspace*{2.65cm} A\_BC\_DIAGONAL$|$B\_AC\_DIAGONAL$|$C\_AB\_DIAGONAL$|$\\
              \hspace*{2.65cm} O\_ABC\_DIAGONAL]}\\
\item{BiasingMethod [UMBRELLA$|$RUIZMONTERO]}\\
\item{BiasingProfile [string]}\\
The name of the file containing the biasing profile.
\item{RuizMonteroFactor [real]}
\item{UmbrellaFactor [real]}\\
The biasing free energy is multiplied by the UmbrellaFactor. This is useful when the biasing free energy goes to infity in certain regions. if the
exact free energy would be used to biased, then the histogram would be flat, even very close to atoms. To keep the repulsion one can lower the used free
energy biasing by e.g.\ multiplying by 0.9.
\item{RestrictMovesToUnitCell [yes$|$no]}\\
Restrict the Monte-Carlo moves to the first unitcell for this component.
\item{BoxAxisABC\_Min [real]}\\
When a particle is restricted in all Monte-Carlo moves (RestrictMovesToUnitCell or RestrictMovesToBox) then do not allow trial moves with a fractional position smaller than BoxAxisABC\_Min.
\item{BoxAxisABC\_Max [real]}\\
When a particle is restricted in all Monte-Carlo moves (RestrictMovesToUnitCell or RestrictMovesToBox) then do not allow trial moves with a fractional position greater than BoxAxisABC\_Max.
\end{itemize}

\subsection*{Transition State Theory settings}
\begin{itemize}
\item{\begin{minipage}{\textwidth}
      FreeEnergyMappingType  [A\_MAPPING$|$B\_MAPPING$|$C\_MAPPING$|$ABC\_MAPPING$|$\\
        \hspace*{4cm}MAP\_AB\_DIAGONAL$|$MAP\_AC\_DIAGONAL$|$MAP\_BC\_DIAGONAL$|$\\
        \hspace*{4cm}MAP\_A\_BC\_DIAGONAL$|$MAP\_B\_AC\_DIAGONAL$|$MAP\_C\_AB\_DIAGONAL$|$\\
        \hspace*{4cm}MAP\_O\_ABC\_DIAGONAL]
      \end{minipage}
      }
Determines how the free energy profile is constructed from the contributions of points in the unit cell.
The free energy is computed using Widom insertion by inserting probe molecules
at many random position inside the unit cell. The `FreeEnergyMappingType' maps a Cartesian position on a reaction coordinate `q'.
The mappings `A\_MAPPING', `B\_MAPPING', `C\_MAPPING' map the Cartesian position onto the `a', `b', `c' lattice vectors.
The diagonal mapping maps onto diagonal, either in 2D or in 3D. For example, `MAP\_A\_BC\_DIAGONAL' maps onto the line from `A' to `B+C' where
`A',`B', and `C' are the end points of the lattive vectors; and `MAP\_O\_ABC\_DIAGONAL' maps onto the line from the origin to the opposite point `A+B+C' on the diagonal.
\item{\begin{minipage}{\textwidth}
      PositionHistogramMappingType  [A\_MAPPING$|$B\_MAPPING$|$C\_MAPPING$|$ABC\_MAPPING$|$\\
        \hspace*{4cm}MAP\_AB\_DIAGONAL$|$MAP\_AC\_DIAGONAL$|$MAP\_BC\_DIAGONAL$|$\\
        \hspace*{4cm}MAP\_A\_BC\_DIAGONAL$|$MAP\_B\_AC\_DIAGONAL$|$MAP\_C\_AB\_DIAGONAL$|$\\
        \hspace*{4cm}MAP\_O\_ABC\_DIAGONAL]
      \end{minipage}
      }
Determines how the position histogram is constructed from the contributions of points in the unit cell.
The free energy is computed from the histogram by using $F(q)=-\log\left[P\left(q\right)\right]$.
The `PositionHistogramMappingType' maps a Cartesian position on a reaction coordinate `q'.
The mappings `A\_MAPPING', `B\_MAPPING', `C\_MAPPING' map the Cartesian position onto the `a', `b', `c' lattice vectors.
The diagonal mapping maps onto diagonal, either in 2D or in 3D. For example, `MAP\_A\_BC\_DIAGONAL' maps onto the line from `A' to `B+C' where
`A',`B', and `C' are the end points of the lattive vectors; and `MAP\_O\_ABC\_DIAGONAL' maps onto the line from the origin to the opposite point `A+B+C' on the diagonal.
\item{PutMoleculeOnBarrier [yes$|$no]}\\
Places the first molecule of component 0 at the position given by `BarrierPosition'. This is used e.g. to start sampling configuration on top of a free energy barrier.
\item{BarrierPosition [real] [real] [real]}\\
The location of the free energy barrier in fractional units of the first unit cell.
\item{MaxBarrierDistance [real]}\\
The maximum distance in \AA ngstrom of the dcTST trajectory.
\item{MaxBarrierTime [real]}\\
The maximum time of the dcTST trajetory in picoseconds.
\item{NumberOfVelocities [int]}\\
The number of times the same initial position of the sampled dcTST starting configurations is used with different initial velocities.
\item{WritedcTSTSnapShotsToFile [yes$|$no]}\\
Whether to write out sampled configuration to a file. The file is stored in the directory `dcTST\_starting\_configurations' and used
as the tarting point to compute the transmission coefficient in dcTST.
\item{WritedcTSTSnapShotsEvery [int]}\\
The frequency in MC cycles of writing out the sampled configurations. Default: 1000.
\end{itemize}